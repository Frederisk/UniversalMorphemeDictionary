\documentclass[10pt, twoside, fontset = none]{ctexbook} % 字型
    \setCJKmainfont{Source Han Serif TC} % 思源中文
    \setCJKsansfont{Source Han Sans TC}
    \setCJKmonofont{Source Han Mono TC}

    \columnseprule = 0.4pt % 分欄線粗細

    \setlength{\parindent}{0em} % 取消縮進
    \setlength{\parskip}{0em} % 取消段間距


% \usepackage[utf8]{inputenc}
% \usepackage[T1]{fontenc}

\usepackage{geometry}
    \geometry{paper = b6j} % B6 Japan paper
    \geometry{columnsep = 10pt}
    \geometry{headsep = 0pt}
    % \geometry{footskip = 0pt}
% 間距和邊距 top=3.5cm,bottom=3.5cm,left=3.7cm,right=4.7cm

\usepackage{fancyhdr} % 頁眉頁腳
    \fancyhf{} % 清除舊格式
    % 頁眉
    \fancyhead[L]{\textsf{\rightmark}}
    \fancyhead[R]{\textsf{\leftmark}}
    \renewcommand{\headrulewidth}{1pt}
    \fancyhead[C]{\textrm{\textbf{\thepage}}}
    % 關閉頁腳
    % \fancyfoot[]{}
    \pagestyle{fancy} % 起效


\usepackage{multicol}

\usepackage{fontspec} % Linux 西文字型
    \setmainfont{Linux Libertine}
    \setsansfont{Linux Biolinum}
    \setmonofont{Linux Libertine Mono}

\usepackage[bf]{titlesec}
    \titleformat{\chapter}
        [display]{\LARGE\sffamily\filcenter\bfseries}
        {\MakeUppercase{\chaptertitlename}\thechapter}
        {0pt}
        {\thispagestyle{empty}}
        []
% 章節標題樣式 bold, sans-serif, centered

\usepackage{paralist}
% 隨機內容

\newenvironment{character}[1]
    {\chapter*{#1}
    \fontsize{6pt}{6pt}\selectfont
    \begin{multicols}{2}}
    {\end{multicols}}

\newenvironment{entry}[2]
    {\hangafter = 1
    \setlength{\hangindent}{1em}
    \markboth{#1}{#1}
    \textbf{#1}{\textsuperscript{<\textit{#2}>}}}
    {\par}


\begin{document}
    % \part*{Cover}
    % \part*{Content}
    \begin{character}{A a}
        \begin{entry}{A}{eng}
            {[美式/ei/;澳洲式/ai/]I. a. \textbf{\textit{n.}}
            \begin{inparaenum}
                \item 英文字母中的第一個,字母A:The first of letter of the word "Apple" is \textit{A}. Apple這個單字的首字母是A。
                \item (由該字母在字母表中的位置衍生而來的)第一個,首個:Item \textit{A} is a number, and item B is also a number. 第一個項目是數字,第二個也是數字。
            \end{inparaenum}
            II. a. \textbf{\textit{n.}}}
            \begin{inparaenum}
                \item (分配字母的等級中的)最高等級,首等;首要,最前方:We assign their levels inspected a rating from A through D, depending on the score.根據得分,我們將他們以等級A到G評級。
            \end{inparaenum}
        \end{entry}
        \begin{entry}{aback}{eng}
            {扉頁是指在書籍封面或襯頁之後、正文之前的一頁。扉頁上一般印有書名、作者或譯者姓名、出版社和出版的年月等。扉頁也起裝飾作用,增加書籍的美觀。}
        \end{entry}

    \end{character}

\end{document}
